\documentclass[12pt]{article}
\usepackage[margin=0.7in]{geometry}

\title{INFO3220 Assignment Stage 2 Report}
\author{Justin Ting, jtin2945, 430203826}
\date{April 2015}

\begin{document}

\maketitle

\textbf{Documentation}\\\\
BaseCodeVersionD did not supply any documentation in the form of a readme file, nor within the source code itself. As such, there were hurdles in setting up the code to run on a machine other than the original coder's. Information that would have been useful to include in the readme would have been:\\
\begin{itemize}
\item \textit{Original build platform} - this would alert a programmer who would later use the code that perhaps some files (specifically, the config.ini where it made an actual difference) would have platform-specific elements, such as Windows line endings in text files. While the original coder was not expect to identify specifically this issue, mentioning Windows as the used platform would have proven helpful.
\item \textit{Use of config.ini} - this would be equally useful for a programmer as it is an end-user - by briefly describing what options are available, how they can be modified, and what behaviour would arise from basic error cases with a corrupted/invalid config.ini file, an end-user would know how much control they have over the program's environment, while a programmer, would know before even looking at code roughly how the config.ini is handled in the source code, and will more quickly interpret its contents once they do delve into it.
\item \textit{Miscellaneous} - while some of the following points don't/necessarily apply to this specific assignment, other information that generally would only be beneficial in a readme include the author, how to report bugs, feature requests, etc., other contact information as necessary, legal notices, etc.
\end{itemize}

\textbf{Extensibility}\\\\
        Image doesn't scale to background size set
        Background and image handled very differently - makes interactions between world elements and the player more troublesome
        Bad consistency - background is a drawn image, player is it's own movie
        Tightly coupled - have to deal with all coordinates for moving
        Had to write a new render method within player class as the /'shortcut' of simply playing a gif movie for an animated character was used

Placeholder text.\\

\textbf{Design}\\\\
The Builder Pattern was used, and I believe that it was a good choice for the
RAII claimed but not actually used (file isn't closed in destructor but in the constructor)
Placeholder text.\\

\textbf{Implementation}\\\\
Placeholder text.\\

\textbf{Style}\\\\
Placeholder text.\\

\end{document}
